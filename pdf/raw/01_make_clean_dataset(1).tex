\documentclass[11pt]{article}

    \usepackage[breakable]{tcolorbox}
    \usepackage{parskip} % Stop auto-indenting (to mimic markdown behaviour)
    
    \usepackage{iftex}
    \ifPDFTeX
    	\usepackage[T1]{fontenc}
    	\usepackage{mathpazo}
    \else
    	\usepackage{fontspec}
    \fi

    % Basic figure setup, for now with no caption control since it's done
    % automatically by Pandoc (which extracts ![](path) syntax from Markdown).
    \usepackage{graphicx}
    % Maintain compatibility with old templates. Remove in nbconvert 6.0
    \let\Oldincludegraphics\includegraphics
    % Ensure that by default, figures have no caption (until we provide a
    % proper Figure object with a Caption API and a way to capture that
    % in the conversion process - todo).
    \usepackage{caption}
    \DeclareCaptionFormat{nocaption}{}
    \captionsetup{format=nocaption,aboveskip=0pt,belowskip=0pt}

    \usepackage{float}
    \floatplacement{figure}{H} % forces figures to be placed at the correct location
    \usepackage{xcolor} % Allow colors to be defined
    \usepackage{enumerate} % Needed for markdown enumerations to work
    \usepackage{geometry} % Used to adjust the document margins
    \usepackage{amsmath} % Equations
    \usepackage{amssymb} % Equations
    \usepackage{textcomp} % defines textquotesingle
    % Hack from http://tex.stackexchange.com/a/47451/13684:
    \AtBeginDocument{%
        \def\PYZsq{\textquotesingle}% Upright quotes in Pygmentized code
    }
    \usepackage{upquote} % Upright quotes for verbatim code
    \usepackage{eurosym} % defines \euro
    \usepackage[mathletters]{ucs} % Extended unicode (utf-8) support
    \usepackage{fancyvrb} % verbatim replacement that allows latex
    \usepackage{grffile} % extends the file name processing of package graphics 
                         % to support a larger range
    \makeatletter % fix for old versions of grffile with XeLaTeX
    \@ifpackagelater{grffile}{2019/11/01}
    {
      % Do nothing on new versions
    }
    {
      \def\Gread@@xetex#1{%
        \IfFileExists{"\Gin@base".bb}%
        {\Gread@eps{\Gin@base.bb}}%
        {\Gread@@xetex@aux#1}%
      }
    }
    \makeatother
    \usepackage[Export]{adjustbox} % Used to constrain images to a maximum size
    \adjustboxset{max size={0.9\linewidth}{0.9\paperheight}}

    % The hyperref package gives us a pdf with properly built
    % internal navigation ('pdf bookmarks' for the table of contents,
    % internal cross-reference links, web links for URLs, etc.)
    \usepackage{hyperref}
    % The default LaTeX title has an obnoxious amount of whitespace. By default,
    % titling removes some of it. It also provides customization options.
    \usepackage{titling}
    \usepackage{longtable} % longtable support required by pandoc >1.10
    \usepackage{booktabs}  % table support for pandoc > 1.12.2
    \usepackage[inline]{enumitem} % IRkernel/repr support (it uses the enumerate* environment)
    \usepackage[normalem]{ulem} % ulem is needed to support strikethroughs (\sout)
                                % normalem makes italics be italics, not underlines
    \usepackage{mathrsfs}
    

    
    % Colors for the hyperref package
    \definecolor{urlcolor}{rgb}{0,.145,.698}
    \definecolor{linkcolor}{rgb}{.71,0.21,0.01}
    \definecolor{citecolor}{rgb}{.12,.54,.11}

    % ANSI colors
    \definecolor{ansi-black}{HTML}{3E424D}
    \definecolor{ansi-black-intense}{HTML}{282C36}
    \definecolor{ansi-red}{HTML}{E75C58}
    \definecolor{ansi-red-intense}{HTML}{B22B31}
    \definecolor{ansi-green}{HTML}{00A250}
    \definecolor{ansi-green-intense}{HTML}{007427}
    \definecolor{ansi-yellow}{HTML}{DDB62B}
    \definecolor{ansi-yellow-intense}{HTML}{B27D12}
    \definecolor{ansi-blue}{HTML}{208FFB}
    \definecolor{ansi-blue-intense}{HTML}{0065CA}
    \definecolor{ansi-magenta}{HTML}{D160C4}
    \definecolor{ansi-magenta-intense}{HTML}{A03196}
    \definecolor{ansi-cyan}{HTML}{60C6C8}
    \definecolor{ansi-cyan-intense}{HTML}{258F8F}
    \definecolor{ansi-white}{HTML}{C5C1B4}
    \definecolor{ansi-white-intense}{HTML}{A1A6B2}
    \definecolor{ansi-default-inverse-fg}{HTML}{FFFFFF}
    \definecolor{ansi-default-inverse-bg}{HTML}{000000}

    % common color for the border for error outputs.
    \definecolor{outerrorbackground}{HTML}{FFDFDF}

    % commands and environments needed by pandoc snippets
    % extracted from the output of `pandoc -s`
    \providecommand{\tightlist}{%
      \setlength{\itemsep}{0pt}\setlength{\parskip}{0pt}}
    \DefineVerbatimEnvironment{Highlighting}{Verbatim}{commandchars=\\\{\}}
    % Add ',fontsize=\small' for more characters per line
    \newenvironment{Shaded}{}{}
    \newcommand{\KeywordTok}[1]{\textcolor[rgb]{0.00,0.44,0.13}{\textbf{{#1}}}}
    \newcommand{\DataTypeTok}[1]{\textcolor[rgb]{0.56,0.13,0.00}{{#1}}}
    \newcommand{\DecValTok}[1]{\textcolor[rgb]{0.25,0.63,0.44}{{#1}}}
    \newcommand{\BaseNTok}[1]{\textcolor[rgb]{0.25,0.63,0.44}{{#1}}}
    \newcommand{\FloatTok}[1]{\textcolor[rgb]{0.25,0.63,0.44}{{#1}}}
    \newcommand{\CharTok}[1]{\textcolor[rgb]{0.25,0.44,0.63}{{#1}}}
    \newcommand{\StringTok}[1]{\textcolor[rgb]{0.25,0.44,0.63}{{#1}}}
    \newcommand{\CommentTok}[1]{\textcolor[rgb]{0.38,0.63,0.69}{\textit{{#1}}}}
    \newcommand{\OtherTok}[1]{\textcolor[rgb]{0.00,0.44,0.13}{{#1}}}
    \newcommand{\AlertTok}[1]{\textcolor[rgb]{1.00,0.00,0.00}{\textbf{{#1}}}}
    \newcommand{\FunctionTok}[1]{\textcolor[rgb]{0.02,0.16,0.49}{{#1}}}
    \newcommand{\RegionMarkerTok}[1]{{#1}}
    \newcommand{\ErrorTok}[1]{\textcolor[rgb]{1.00,0.00,0.00}{\textbf{{#1}}}}
    \newcommand{\NormalTok}[1]{{#1}}
    
    % Additional commands for more recent versions of Pandoc
    \newcommand{\ConstantTok}[1]{\textcolor[rgb]{0.53,0.00,0.00}{{#1}}}
    \newcommand{\SpecialCharTok}[1]{\textcolor[rgb]{0.25,0.44,0.63}{{#1}}}
    \newcommand{\VerbatimStringTok}[1]{\textcolor[rgb]{0.25,0.44,0.63}{{#1}}}
    \newcommand{\SpecialStringTok}[1]{\textcolor[rgb]{0.73,0.40,0.53}{{#1}}}
    \newcommand{\ImportTok}[1]{{#1}}
    \newcommand{\DocumentationTok}[1]{\textcolor[rgb]{0.73,0.13,0.13}{\textit{{#1}}}}
    \newcommand{\AnnotationTok}[1]{\textcolor[rgb]{0.38,0.63,0.69}{\textbf{\textit{{#1}}}}}
    \newcommand{\CommentVarTok}[1]{\textcolor[rgb]{0.38,0.63,0.69}{\textbf{\textit{{#1}}}}}
    \newcommand{\VariableTok}[1]{\textcolor[rgb]{0.10,0.09,0.49}{{#1}}}
    \newcommand{\ControlFlowTok}[1]{\textcolor[rgb]{0.00,0.44,0.13}{\textbf{{#1}}}}
    \newcommand{\OperatorTok}[1]{\textcolor[rgb]{0.40,0.40,0.40}{{#1}}}
    \newcommand{\BuiltInTok}[1]{{#1}}
    \newcommand{\ExtensionTok}[1]{{#1}}
    \newcommand{\PreprocessorTok}[1]{\textcolor[rgb]{0.74,0.48,0.00}{{#1}}}
    \newcommand{\AttributeTok}[1]{\textcolor[rgb]{0.49,0.56,0.16}{{#1}}}
    \newcommand{\InformationTok}[1]{\textcolor[rgb]{0.38,0.63,0.69}{\textbf{\textit{{#1}}}}}
    \newcommand{\WarningTok}[1]{\textcolor[rgb]{0.38,0.63,0.69}{\textbf{\textit{{#1}}}}}
    
    
    % Define a nice break command that doesn't care if a line doesn't already
    % exist.
    \def\br{\hspace*{\fill} \\* }
    % Math Jax compatibility definitions
    \def\gt{>}
    \def\lt{<}
    \let\Oldtex\TeX
    \let\Oldlatex\LaTeX
    \renewcommand{\TeX}{\textrm{\Oldtex}}
    \renewcommand{\LaTeX}{\textrm{\Oldlatex}}
    % Document parameters
    % Document title
    \title{01\_make\_clean\_dataset}
    
    
    
    
    
% Pygments definitions
\makeatletter
\def\PY@reset{\let\PY@it=\relax \let\PY@bf=\relax%
    \let\PY@ul=\relax \let\PY@tc=\relax%
    \let\PY@bc=\relax \let\PY@ff=\relax}
\def\PY@tok#1{\csname PY@tok@#1\endcsname}
\def\PY@toks#1+{\ifx\relax#1\empty\else%
    \PY@tok{#1}\expandafter\PY@toks\fi}
\def\PY@do#1{\PY@bc{\PY@tc{\PY@ul{%
    \PY@it{\PY@bf{\PY@ff{#1}}}}}}}
\def\PY#1#2{\PY@reset\PY@toks#1+\relax+\PY@do{#2}}

\expandafter\def\csname PY@tok@w\endcsname{\def\PY@tc##1{\textcolor[rgb]{0.73,0.73,0.73}{##1}}}
\expandafter\def\csname PY@tok@c\endcsname{\let\PY@it=\textit\def\PY@tc##1{\textcolor[rgb]{0.25,0.50,0.50}{##1}}}
\expandafter\def\csname PY@tok@cp\endcsname{\def\PY@tc##1{\textcolor[rgb]{0.74,0.48,0.00}{##1}}}
\expandafter\def\csname PY@tok@k\endcsname{\let\PY@bf=\textbf\def\PY@tc##1{\textcolor[rgb]{0.00,0.50,0.00}{##1}}}
\expandafter\def\csname PY@tok@kp\endcsname{\def\PY@tc##1{\textcolor[rgb]{0.00,0.50,0.00}{##1}}}
\expandafter\def\csname PY@tok@kt\endcsname{\def\PY@tc##1{\textcolor[rgb]{0.69,0.00,0.25}{##1}}}
\expandafter\def\csname PY@tok@o\endcsname{\def\PY@tc##1{\textcolor[rgb]{0.40,0.40,0.40}{##1}}}
\expandafter\def\csname PY@tok@ow\endcsname{\let\PY@bf=\textbf\def\PY@tc##1{\textcolor[rgb]{0.67,0.13,1.00}{##1}}}
\expandafter\def\csname PY@tok@nb\endcsname{\def\PY@tc##1{\textcolor[rgb]{0.00,0.50,0.00}{##1}}}
\expandafter\def\csname PY@tok@nf\endcsname{\def\PY@tc##1{\textcolor[rgb]{0.00,0.00,1.00}{##1}}}
\expandafter\def\csname PY@tok@nc\endcsname{\let\PY@bf=\textbf\def\PY@tc##1{\textcolor[rgb]{0.00,0.00,1.00}{##1}}}
\expandafter\def\csname PY@tok@nn\endcsname{\let\PY@bf=\textbf\def\PY@tc##1{\textcolor[rgb]{0.00,0.00,1.00}{##1}}}
\expandafter\def\csname PY@tok@ne\endcsname{\let\PY@bf=\textbf\def\PY@tc##1{\textcolor[rgb]{0.82,0.25,0.23}{##1}}}
\expandafter\def\csname PY@tok@nv\endcsname{\def\PY@tc##1{\textcolor[rgb]{0.10,0.09,0.49}{##1}}}
\expandafter\def\csname PY@tok@no\endcsname{\def\PY@tc##1{\textcolor[rgb]{0.53,0.00,0.00}{##1}}}
\expandafter\def\csname PY@tok@nl\endcsname{\def\PY@tc##1{\textcolor[rgb]{0.63,0.63,0.00}{##1}}}
\expandafter\def\csname PY@tok@ni\endcsname{\let\PY@bf=\textbf\def\PY@tc##1{\textcolor[rgb]{0.60,0.60,0.60}{##1}}}
\expandafter\def\csname PY@tok@na\endcsname{\def\PY@tc##1{\textcolor[rgb]{0.49,0.56,0.16}{##1}}}
\expandafter\def\csname PY@tok@nt\endcsname{\let\PY@bf=\textbf\def\PY@tc##1{\textcolor[rgb]{0.00,0.50,0.00}{##1}}}
\expandafter\def\csname PY@tok@nd\endcsname{\def\PY@tc##1{\textcolor[rgb]{0.67,0.13,1.00}{##1}}}
\expandafter\def\csname PY@tok@s\endcsname{\def\PY@tc##1{\textcolor[rgb]{0.73,0.13,0.13}{##1}}}
\expandafter\def\csname PY@tok@sd\endcsname{\let\PY@it=\textit\def\PY@tc##1{\textcolor[rgb]{0.73,0.13,0.13}{##1}}}
\expandafter\def\csname PY@tok@si\endcsname{\let\PY@bf=\textbf\def\PY@tc##1{\textcolor[rgb]{0.73,0.40,0.53}{##1}}}
\expandafter\def\csname PY@tok@se\endcsname{\let\PY@bf=\textbf\def\PY@tc##1{\textcolor[rgb]{0.73,0.40,0.13}{##1}}}
\expandafter\def\csname PY@tok@sr\endcsname{\def\PY@tc##1{\textcolor[rgb]{0.73,0.40,0.53}{##1}}}
\expandafter\def\csname PY@tok@ss\endcsname{\def\PY@tc##1{\textcolor[rgb]{0.10,0.09,0.49}{##1}}}
\expandafter\def\csname PY@tok@sx\endcsname{\def\PY@tc##1{\textcolor[rgb]{0.00,0.50,0.00}{##1}}}
\expandafter\def\csname PY@tok@m\endcsname{\def\PY@tc##1{\textcolor[rgb]{0.40,0.40,0.40}{##1}}}
\expandafter\def\csname PY@tok@gh\endcsname{\let\PY@bf=\textbf\def\PY@tc##1{\textcolor[rgb]{0.00,0.00,0.50}{##1}}}
\expandafter\def\csname PY@tok@gu\endcsname{\let\PY@bf=\textbf\def\PY@tc##1{\textcolor[rgb]{0.50,0.00,0.50}{##1}}}
\expandafter\def\csname PY@tok@gd\endcsname{\def\PY@tc##1{\textcolor[rgb]{0.63,0.00,0.00}{##1}}}
\expandafter\def\csname PY@tok@gi\endcsname{\def\PY@tc##1{\textcolor[rgb]{0.00,0.63,0.00}{##1}}}
\expandafter\def\csname PY@tok@gr\endcsname{\def\PY@tc##1{\textcolor[rgb]{1.00,0.00,0.00}{##1}}}
\expandafter\def\csname PY@tok@ge\endcsname{\let\PY@it=\textit}
\expandafter\def\csname PY@tok@gs\endcsname{\let\PY@bf=\textbf}
\expandafter\def\csname PY@tok@gp\endcsname{\let\PY@bf=\textbf\def\PY@tc##1{\textcolor[rgb]{0.00,0.00,0.50}{##1}}}
\expandafter\def\csname PY@tok@go\endcsname{\def\PY@tc##1{\textcolor[rgb]{0.53,0.53,0.53}{##1}}}
\expandafter\def\csname PY@tok@gt\endcsname{\def\PY@tc##1{\textcolor[rgb]{0.00,0.27,0.87}{##1}}}
\expandafter\def\csname PY@tok@err\endcsname{\def\PY@bc##1{\setlength{\fboxsep}{0pt}\fcolorbox[rgb]{1.00,0.00,0.00}{1,1,1}{\strut ##1}}}
\expandafter\def\csname PY@tok@kc\endcsname{\let\PY@bf=\textbf\def\PY@tc##1{\textcolor[rgb]{0.00,0.50,0.00}{##1}}}
\expandafter\def\csname PY@tok@kd\endcsname{\let\PY@bf=\textbf\def\PY@tc##1{\textcolor[rgb]{0.00,0.50,0.00}{##1}}}
\expandafter\def\csname PY@tok@kn\endcsname{\let\PY@bf=\textbf\def\PY@tc##1{\textcolor[rgb]{0.00,0.50,0.00}{##1}}}
\expandafter\def\csname PY@tok@kr\endcsname{\let\PY@bf=\textbf\def\PY@tc##1{\textcolor[rgb]{0.00,0.50,0.00}{##1}}}
\expandafter\def\csname PY@tok@bp\endcsname{\def\PY@tc##1{\textcolor[rgb]{0.00,0.50,0.00}{##1}}}
\expandafter\def\csname PY@tok@fm\endcsname{\def\PY@tc##1{\textcolor[rgb]{0.00,0.00,1.00}{##1}}}
\expandafter\def\csname PY@tok@vc\endcsname{\def\PY@tc##1{\textcolor[rgb]{0.10,0.09,0.49}{##1}}}
\expandafter\def\csname PY@tok@vg\endcsname{\def\PY@tc##1{\textcolor[rgb]{0.10,0.09,0.49}{##1}}}
\expandafter\def\csname PY@tok@vi\endcsname{\def\PY@tc##1{\textcolor[rgb]{0.10,0.09,0.49}{##1}}}
\expandafter\def\csname PY@tok@vm\endcsname{\def\PY@tc##1{\textcolor[rgb]{0.10,0.09,0.49}{##1}}}
\expandafter\def\csname PY@tok@sa\endcsname{\def\PY@tc##1{\textcolor[rgb]{0.73,0.13,0.13}{##1}}}
\expandafter\def\csname PY@tok@sb\endcsname{\def\PY@tc##1{\textcolor[rgb]{0.73,0.13,0.13}{##1}}}
\expandafter\def\csname PY@tok@sc\endcsname{\def\PY@tc##1{\textcolor[rgb]{0.73,0.13,0.13}{##1}}}
\expandafter\def\csname PY@tok@dl\endcsname{\def\PY@tc##1{\textcolor[rgb]{0.73,0.13,0.13}{##1}}}
\expandafter\def\csname PY@tok@s2\endcsname{\def\PY@tc##1{\textcolor[rgb]{0.73,0.13,0.13}{##1}}}
\expandafter\def\csname PY@tok@sh\endcsname{\def\PY@tc##1{\textcolor[rgb]{0.73,0.13,0.13}{##1}}}
\expandafter\def\csname PY@tok@s1\endcsname{\def\PY@tc##1{\textcolor[rgb]{0.73,0.13,0.13}{##1}}}
\expandafter\def\csname PY@tok@mb\endcsname{\def\PY@tc##1{\textcolor[rgb]{0.40,0.40,0.40}{##1}}}
\expandafter\def\csname PY@tok@mf\endcsname{\def\PY@tc##1{\textcolor[rgb]{0.40,0.40,0.40}{##1}}}
\expandafter\def\csname PY@tok@mh\endcsname{\def\PY@tc##1{\textcolor[rgb]{0.40,0.40,0.40}{##1}}}
\expandafter\def\csname PY@tok@mi\endcsname{\def\PY@tc##1{\textcolor[rgb]{0.40,0.40,0.40}{##1}}}
\expandafter\def\csname PY@tok@il\endcsname{\def\PY@tc##1{\textcolor[rgb]{0.40,0.40,0.40}{##1}}}
\expandafter\def\csname PY@tok@mo\endcsname{\def\PY@tc##1{\textcolor[rgb]{0.40,0.40,0.40}{##1}}}
\expandafter\def\csname PY@tok@ch\endcsname{\let\PY@it=\textit\def\PY@tc##1{\textcolor[rgb]{0.25,0.50,0.50}{##1}}}
\expandafter\def\csname PY@tok@cm\endcsname{\let\PY@it=\textit\def\PY@tc##1{\textcolor[rgb]{0.25,0.50,0.50}{##1}}}
\expandafter\def\csname PY@tok@cpf\endcsname{\let\PY@it=\textit\def\PY@tc##1{\textcolor[rgb]{0.25,0.50,0.50}{##1}}}
\expandafter\def\csname PY@tok@c1\endcsname{\let\PY@it=\textit\def\PY@tc##1{\textcolor[rgb]{0.25,0.50,0.50}{##1}}}
\expandafter\def\csname PY@tok@cs\endcsname{\let\PY@it=\textit\def\PY@tc##1{\textcolor[rgb]{0.25,0.50,0.50}{##1}}}

\def\PYZbs{\char`\\}
\def\PYZus{\char`\_}
\def\PYZob{\char`\{}
\def\PYZcb{\char`\}}
\def\PYZca{\char`\^}
\def\PYZam{\char`\&}
\def\PYZlt{\char`\<}
\def\PYZgt{\char`\>}
\def\PYZsh{\char`\#}
\def\PYZpc{\char`\%}
\def\PYZdl{\char`\$}
\def\PYZhy{\char`\-}
\def\PYZsq{\char`\'}
\def\PYZdq{\char`\"}
\def\PYZti{\char`\~}
% for compatibility with earlier versions
\def\PYZat{@}
\def\PYZlb{[}
\def\PYZrb{]}
\makeatother


    % For linebreaks inside Verbatim environment from package fancyvrb. 
    \makeatletter
        \newbox\Wrappedcontinuationbox 
        \newbox\Wrappedvisiblespacebox 
        \newcommand*\Wrappedvisiblespace {\textcolor{red}{\textvisiblespace}} 
        \newcommand*\Wrappedcontinuationsymbol {\textcolor{red}{\llap{\tiny$\m@th\hookrightarrow$}}} 
        \newcommand*\Wrappedcontinuationindent {3ex } 
        \newcommand*\Wrappedafterbreak {\kern\Wrappedcontinuationindent\copy\Wrappedcontinuationbox} 
        % Take advantage of the already applied Pygments mark-up to insert 
        % potential linebreaks for TeX processing. 
        %        {, <, #, %, $, ' and ": go to next line. 
        %        _, }, ^, &, >, - and ~: stay at end of broken line. 
        % Use of \textquotesingle for straight quote. 
        \newcommand*\Wrappedbreaksatspecials {% 
            \def\PYGZus{\discretionary{\char`\_}{\Wrappedafterbreak}{\char`\_}}% 
            \def\PYGZob{\discretionary{}{\Wrappedafterbreak\char`\{}{\char`\{}}% 
            \def\PYGZcb{\discretionary{\char`\}}{\Wrappedafterbreak}{\char`\}}}% 
            \def\PYGZca{\discretionary{\char`\^}{\Wrappedafterbreak}{\char`\^}}% 
            \def\PYGZam{\discretionary{\char`\&}{\Wrappedafterbreak}{\char`\&}}% 
            \def\PYGZlt{\discretionary{}{\Wrappedafterbreak\char`\<}{\char`\<}}% 
            \def\PYGZgt{\discretionary{\char`\>}{\Wrappedafterbreak}{\char`\>}}% 
            \def\PYGZsh{\discretionary{}{\Wrappedafterbreak\char`\#}{\char`\#}}% 
            \def\PYGZpc{\discretionary{}{\Wrappedafterbreak\char`\%}{\char`\%}}% 
            \def\PYGZdl{\discretionary{}{\Wrappedafterbreak\char`\$}{\char`\$}}% 
            \def\PYGZhy{\discretionary{\char`\-}{\Wrappedafterbreak}{\char`\-}}% 
            \def\PYGZsq{\discretionary{}{\Wrappedafterbreak\textquotesingle}{\textquotesingle}}% 
            \def\PYGZdq{\discretionary{}{\Wrappedafterbreak\char`\"}{\char`\"}}% 
            \def\PYGZti{\discretionary{\char`\~}{\Wrappedafterbreak}{\char`\~}}% 
        } 
        % Some characters . , ; ? ! / are not pygmentized. 
        % This macro makes them "active" and they will insert potential linebreaks 
        \newcommand*\Wrappedbreaksatpunct {% 
            \lccode`\~`\.\lowercase{\def~}{\discretionary{\hbox{\char`\.}}{\Wrappedafterbreak}{\hbox{\char`\.}}}% 
            \lccode`\~`\,\lowercase{\def~}{\discretionary{\hbox{\char`\,}}{\Wrappedafterbreak}{\hbox{\char`\,}}}% 
            \lccode`\~`\;\lowercase{\def~}{\discretionary{\hbox{\char`\;}}{\Wrappedafterbreak}{\hbox{\char`\;}}}% 
            \lccode`\~`\:\lowercase{\def~}{\discretionary{\hbox{\char`\:}}{\Wrappedafterbreak}{\hbox{\char`\:}}}% 
            \lccode`\~`\?\lowercase{\def~}{\discretionary{\hbox{\char`\?}}{\Wrappedafterbreak}{\hbox{\char`\?}}}% 
            \lccode`\~`\!\lowercase{\def~}{\discretionary{\hbox{\char`\!}}{\Wrappedafterbreak}{\hbox{\char`\!}}}% 
            \lccode`\~`\/\lowercase{\def~}{\discretionary{\hbox{\char`\/}}{\Wrappedafterbreak}{\hbox{\char`\/}}}% 
            \catcode`\.\active
            \catcode`\,\active 
            \catcode`\;\active
            \catcode`\:\active
            \catcode`\?\active
            \catcode`\!\active
            \catcode`\/\active 
            \lccode`\~`\~ 	
        }
    \makeatother

    \let\OriginalVerbatim=\Verbatim
    \makeatletter
    \renewcommand{\Verbatim}[1][1]{%
        %\parskip\z@skip
        \sbox\Wrappedcontinuationbox {\Wrappedcontinuationsymbol}%
        \sbox\Wrappedvisiblespacebox {\FV@SetupFont\Wrappedvisiblespace}%
        \def\FancyVerbFormatLine ##1{\hsize\linewidth
            \vtop{\raggedright\hyphenpenalty\z@\exhyphenpenalty\z@
                \doublehyphendemerits\z@\finalhyphendemerits\z@
                \strut ##1\strut}%
        }%
        % If the linebreak is at a space, the latter will be displayed as visible
        % space at end of first line, and a continuation symbol starts next line.
        % Stretch/shrink are however usually zero for typewriter font.
        \def\FV@Space {%
            \nobreak\hskip\z@ plus\fontdimen3\font minus\fontdimen4\font
            \discretionary{\copy\Wrappedvisiblespacebox}{\Wrappedafterbreak}
            {\kern\fontdimen2\font}%
        }%
        
        % Allow breaks at special characters using \PYG... macros.
        \Wrappedbreaksatspecials
        % Breaks at punctuation characters . , ; ? ! and / need catcode=\active 	
        \OriginalVerbatim[#1,codes*=\Wrappedbreaksatpunct]%
    }
    \makeatother

    % Exact colors from NB
    \definecolor{incolor}{HTML}{303F9F}
    \definecolor{outcolor}{HTML}{D84315}
    \definecolor{cellborder}{HTML}{CFCFCF}
    \definecolor{cellbackground}{HTML}{F7F7F7}
    
    % prompt
    \makeatletter
    \newcommand{\boxspacing}{\kern\kvtcb@left@rule\kern\kvtcb@boxsep}
    \makeatother
    \newcommand{\prompt}[4]{
        {\ttfamily\llap{{\color{#2}[#3]:\hspace{3pt}#4}}\vspace{-\baselineskip}}
    }
    

    
    % Prevent overflowing lines due to hard-to-break entities
    \sloppy 
    % Setup hyperref package
    \hypersetup{
      breaklinks=true,  % so long urls are correctly broken across lines
      colorlinks=true,
      urlcolor=urlcolor,
      linkcolor=linkcolor,
      citecolor=citecolor,
      }
    % Slightly bigger margins than the latex defaults
    
    \geometry{verbose,tmargin=1in,bmargin=1in,lmargin=1in,rmargin=1in}
    
    

\begin{document}
    
    \maketitle
    
    

    
    In this notebook we will import the raw data from the WGM, clean them
and dummy-code them in order to make them compatible with the future
analysis we will run.

    \begin{tcolorbox}[breakable, size=fbox, boxrule=1pt, pad at break*=1mm,colback=cellbackground, colframe=cellborder]
\prompt{In}{incolor}{ }{\boxspacing}
\begin{Verbatim}[commandchars=\\\{\}]

\end{Verbatim}
\end{tcolorbox}

    \hypertarget{import-the-packages}{%
\section{Import the packages}\label{import-the-packages}}

    \begin{tcolorbox}[breakable, size=fbox, boxrule=1pt, pad at break*=1mm,colback=cellbackground, colframe=cellborder]
\prompt{In}{incolor}{3}{\boxspacing}
\begin{Verbatim}[commandchars=\\\{\}]
\PY{k+kn}{import} \PY{n+nn}{pandas} \PY{k}{as} \PY{n+nn}{pd}
\PY{k+kn}{import} \PY{n+nn}{numpy} \PY{k}{as} \PY{n+nn}{np}
\PY{k+kn}{import} \PY{n+nn}{scipy}\PY{n+nn}{.}\PY{n+nn}{stats} \PY{k}{as} \PY{n+nn}{stt}
\PY{k+kn}{import} \PY{n+nn}{networkx} \PY{k}{as} \PY{n+nn}{nx}
\PY{k+kn}{import} \PY{n+nn}{matplotlib}\PY{n+nn}{.}\PY{n+nn}{pyplot} \PY{k}{as} \PY{n+nn}{plt}
\end{Verbatim}
\end{tcolorbox}

    \begin{tcolorbox}[breakable, size=fbox, boxrule=1pt, pad at break*=1mm,colback=cellbackground, colframe=cellborder]
\prompt{In}{incolor}{ }{\boxspacing}
\begin{Verbatim}[commandchars=\\\{\}]

\end{Verbatim}
\end{tcolorbox}

    \hypertarget{import-the-raw-data}{%
\section{Import the raw data}\label{import-the-raw-data}}

    Main data:

\texttt{wgm\_raw} the full database

\texttt{wgm\_dic} a dictionary of what the database means
-\textgreater{} the important columns are the \texttt{code},
\texttt{long\ question} and \texttt{short\ question}

Note: wgm\_dic is not a dictionary data type, but a dataframe. This has
been done as we need to convert between 3 different types of dataframe
we will deal with:

\begin{itemize}
\tightlist
\item
  boolean (i.e.~dummy coded)
\item
  labels (i.e.~very entry is
\item
  numeric
\end{itemize}

    The file \texttt{wgm2018.xlsx} is the raw file provided by the Wellcome
Global Monitor:
https://wellcome.org/reports/wellcome-global-monitor/2018

Instead, the \texttt{wgm2018\_data\_dic\_mod.xlsx} is a file made by us
to rename the questions and the answers in a more compact way for when
dummy coding. You can find it here:
https://github.com/just-a-normal-dino/wgm18\_dic

    \begin{tcolorbox}[breakable, size=fbox, boxrule=1pt, pad at break*=1mm,colback=cellbackground, colframe=cellborder]
\prompt{In}{incolor}{4}{\boxspacing}
\begin{Verbatim}[commandchars=\\\{\}]
\PY{c+c1}{\PYZsh{} Import the raw data}
\PY{n}{wgm\PYZus{}raw} \PY{o}{=} \PY{n}{pd}\PY{o}{.}\PY{n}{read\PYZus{}excel}\PY{p}{(}\PY{l+s+s1}{\PYZsq{}}\PY{l+s+s1}{wgm2018.xlsx}\PY{l+s+s1}{\PYZsq{}}\PY{p}{,} \PY{n}{sheet\PYZus{}name}\PY{o}{=}\PY{l+m+mi}{1}\PY{p}{)}
\PY{n}{wgm\PYZus{}dic} \PY{o}{=} \PY{n}{pd}\PY{o}{.}\PY{n}{read\PYZus{}excel}\PY{p}{(}\PY{l+s+s1}{\PYZsq{}}\PY{l+s+s1}{wgm2018\PYZus{}data\PYZus{}dic\PYZus{}mod.xlsx}\PY{l+s+s1}{\PYZsq{}}\PY{p}{)}
\PY{n}{wgm} \PY{o}{=} \PY{n}{wgm\PYZus{}raw}\PY{o}{.}\PY{n}{copy}\PY{p}{(}\PY{p}{)}
\end{Verbatim}
\end{tcolorbox}

    Display the raw data

    \begin{tcolorbox}[breakable, size=fbox, boxrule=1pt, pad at break*=1mm,colback=cellbackground, colframe=cellborder]
\prompt{In}{incolor}{5}{\boxspacing}
\begin{Verbatim}[commandchars=\\\{\}]
\PY{c+c1}{\PYZsh{} wgm\PYZus{}raw.info()}
\PY{n}{wgm}\PY{o}{.}\PY{n}{head}\PY{p}{(}\PY{p}{)}
\end{Verbatim}
\end{tcolorbox}

            \begin{tcolorbox}[breakable, size=fbox, boxrule=.5pt, pad at break*=1mm, opacityfill=0]
\prompt{Out}{outcolor}{5}{\boxspacing}
\begin{Verbatim}[commandchars=\\\{\}]
   WP5       wgt         PROJWT FIELD\_DATE  YEAR\_CALENDAR  Q1  Q2  Q3  Q4  \textbackslash{}
0    1  0.652821  171769.597742 2018-01-08           2018   3   2   1   2
1    1  0.695706  183053.484155 2018-01-08           2018   2   2   1   2
2    1  0.523829  137829.328857 2018-01-08           2018   2   2   1  98
3    1  0.764442  201139.215039 2018-01-08           2018   2   1   1   2
4    1  3.327946  875645.512738 2018-01-08           2018   2   1   1   2

   Q5A  {\ldots} Age AgeCategories  Gender  Education  Urban\_Rural  \textbackslash{}
0    2  {\ldots}  72             3       2          3            1
1    1  {\ldots}  72             3       1          2            2
2    1  {\ldots}  85             3       1          2            1
3    1  {\ldots}  54             3       1          3            2
4    1  {\ldots}  20             1       1          2            2

   Household\_Income  Regions\_Report  Subjective\_Income  WBI EMP\_2010
0                 3               7                  2    4        6
1                 3               7                  1    4        6
2                 2               7                  3    4        6
3                 5               7                  1    4        1
4                 2               7                  1    4        6

[5 rows x 60 columns]
\end{Verbatim}
\end{tcolorbox}
        
    Display the dictionary

    \begin{tcolorbox}[breakable, size=fbox, boxrule=1pt, pad at break*=1mm,colback=cellbackground, colframe=cellborder]
\prompt{In}{incolor}{7}{\boxspacing}
\begin{Verbatim}[commandchars=\\\{\}]
\PY{n}{wgm\PYZus{}dic}\PY{o}{.}\PY{n}{head}\PY{p}{(}\PY{p}{)}
\end{Verbatim}
\end{tcolorbox}

            \begin{tcolorbox}[breakable, size=fbox, boxrule=.5pt, pad at break*=1mm, opacityfill=0]
\prompt{Out}{outcolor}{7}{\boxspacing}
\begin{Verbatim}[commandchars=\\\{\}]
            Code                                      Long question  \textbackslash{}
0            WP5                                            Country
1            wgt  National weight, for analysis at the country l{\ldots}
2         PROJWT  Population weight (included factor to project {\ldots}
3     FIELD\_DATE                              Study Completion Date
4  YEAR\_CALENDAR                                    Year of survey

    Short question  Trust in science value  \textbackslash{}
0          Country                       0
1       Nat weight                       0
2       Pop weight                       0
3  Completion Date                       0
4      Survey Year                       0

                                             Ans dic  \textbackslash{}
0  1=United States, 2=Egypt, 3=Morocco, 4=Lebanon{\ldots}
1                            Scale (value of weight)
2                            Scale (value of weight)
3                                               Date
4                                               Year

                                               Notes
0                                                NaN
1  Use this weight for analysis at the country level
2  Use this weight for analysis which pools toget{\ldots}
3                                                NaN
4                                                NaN
\end{Verbatim}
\end{tcolorbox}
        
    \begin{tcolorbox}[breakable, size=fbox, boxrule=1pt, pad at break*=1mm,colback=cellbackground, colframe=cellborder]
\prompt{In}{incolor}{ }{\boxspacing}
\begin{Verbatim}[commandchars=\\\{\}]

\end{Verbatim}
\end{tcolorbox}

    \begin{center}\rule{0.5\linewidth}{0.5pt}\end{center}

    \begin{tcolorbox}[breakable, size=fbox, boxrule=1pt, pad at break*=1mm,colback=cellbackground, colframe=cellborder]
\prompt{In}{incolor}{ }{\boxspacing}
\begin{Verbatim}[commandchars=\\\{\}]

\end{Verbatim}
\end{tcolorbox}

    \hypertarget{clean-the-dictionary}{%
\section{Clean the dictionary}\label{clean-the-dictionary}}

    Drop the notes column

    \begin{tcolorbox}[breakable, size=fbox, boxrule=1pt, pad at break*=1mm,colback=cellbackground, colframe=cellborder]
\prompt{In}{incolor}{8}{\boxspacing}
\begin{Verbatim}[commandchars=\\\{\}]
\PY{c+c1}{\PYZsh{} Note: if you\PYZsq{}ll run this cell twice, you\PYZsq{}ll get an error as it cannot delete it twice}
\PY{n}{wgm\PYZus{}dic}\PY{o}{.}\PY{n}{drop}\PY{p}{(}\PY{n}{columns}\PY{o}{=}\PY{l+s+s2}{\PYZdq{}}\PY{l+s+s2}{Notes}\PY{l+s+s2}{\PYZdq{}}\PY{p}{,} \PY{n}{inplace}\PY{o}{=}\PY{k+kc}{True}\PY{p}{)}
\PY{n}{wgm\PYZus{}dic}\PY{o}{.}\PY{n}{head}\PY{p}{(}\PY{p}{)}
\end{Verbatim}
\end{tcolorbox}

            \begin{tcolorbox}[breakable, size=fbox, boxrule=.5pt, pad at break*=1mm, opacityfill=0]
\prompt{Out}{outcolor}{8}{\boxspacing}
\begin{Verbatim}[commandchars=\\\{\}]
            Code                                      Long question  \textbackslash{}
0            WP5                                            Country
1            wgt  National weight, for analysis at the country l{\ldots}
2         PROJWT  Population weight (included factor to project {\ldots}
3     FIELD\_DATE                              Study Completion Date
4  YEAR\_CALENDAR                                    Year of survey

    Short question  Trust in science value  \textbackslash{}
0          Country                       0
1       Nat weight                       0
2       Pop weight                       0
3  Completion Date                       0
4      Survey Year                       0

                                             Ans dic
0  1=United States, 2=Egypt, 3=Morocco, 4=Lebanon{\ldots}
1                            Scale (value of weight)
2                            Scale (value of weight)
3                                               Date
4                                               Year
\end{Verbatim}
\end{tcolorbox}
        
    Make the \texttt{code} column as the index of the dictionary (and
duplicate it so I can easily access it as a column)

    \begin{tcolorbox}[breakable, size=fbox, boxrule=1pt, pad at break*=1mm,colback=cellbackground, colframe=cellborder]
\prompt{In}{incolor}{9}{\boxspacing}
\begin{Verbatim}[commandchars=\\\{\}]
\PY{n}{wgm\PYZus{}dic}\PY{p}{[}\PY{l+s+s2}{\PYZdq{}}\PY{l+s+s2}{Code\PYZus{}i}\PY{l+s+s2}{\PYZdq{}}\PY{p}{]} \PY{o}{=} \PY{n}{wgm\PYZus{}dic}\PY{p}{[}\PY{l+s+s2}{\PYZdq{}}\PY{l+s+s2}{Code}\PY{l+s+s2}{\PYZdq{}}\PY{p}{]}
\PY{n}{wgm\PYZus{}dic}\PY{o}{.}\PY{n}{set\PYZus{}index}\PY{p}{(}\PY{l+s+s2}{\PYZdq{}}\PY{l+s+s2}{Code\PYZus{}i}\PY{l+s+s2}{\PYZdq{}}\PY{p}{,}\PY{n}{inplace}\PY{o}{=}\PY{k+kc}{True}\PY{p}{)}
\PY{n}{wgm\PYZus{}dic}\PY{o}{.}\PY{n}{head}\PY{p}{(}\PY{p}{)}
\end{Verbatim}
\end{tcolorbox}

            \begin{tcolorbox}[breakable, size=fbox, boxrule=.5pt, pad at break*=1mm, opacityfill=0]
\prompt{Out}{outcolor}{9}{\boxspacing}
\begin{Verbatim}[commandchars=\\\{\}]
                        Code  \textbackslash{}
Code\_i
WP5                      WP5
wgt                      wgt
PROJWT                PROJWT
FIELD\_DATE        FIELD\_DATE
YEAR\_CALENDAR  YEAR\_CALENDAR

                                                   Long question  \textbackslash{}
Code\_i
WP5                                                      Country
wgt            National weight, for analysis at the country l{\ldots}
PROJWT         Population weight (included factor to project {\ldots}
FIELD\_DATE                                 Study Completion Date
YEAR\_CALENDAR                                    Year of survey

                Short question  Trust in science value  \textbackslash{}
Code\_i
WP5                    Country                       0
wgt                 Nat weight                       0
PROJWT              Pop weight                       0
FIELD\_DATE     Completion Date                       0
YEAR\_CALENDAR      Survey Year                       0

                                                         Ans dic
Code\_i
WP5            1=United States, 2=Egypt, 3=Morocco, 4=Lebanon{\ldots}
wgt                                      Scale (value of weight)
PROJWT                                   Scale (value of weight)
FIELD\_DATE                                                  Date
YEAR\_CALENDAR                                               Year
\end{Verbatim}
\end{tcolorbox}
        
    Add a new columns which tells you if the value is a cathegory or not
(\texttt{Categorical\ Ans}). This would be true if the answers are
categorical (aka ``nominal''). And it would be false for continuous
numeric variables such as age.

    \begin{tcolorbox}[breakable, size=fbox, boxrule=1pt, pad at break*=1mm,colback=cellbackground, colframe=cellborder]
\prompt{In}{incolor}{11}{\boxspacing}
\begin{Verbatim}[commandchars=\\\{\}]
\PY{n}{ans\PYZus{}col} \PY{o}{=} \PY{n}{wgm\PYZus{}dic}\PY{p}{[}\PY{l+s+s2}{\PYZdq{}}\PY{l+s+s2}{Ans dic}\PY{l+s+s2}{\PYZdq{}}\PY{p}{]}
\PY{n}{is\PYZus{}category} \PY{o}{=} \PY{n}{ans\PYZus{}col}\PY{o}{.}\PY{n}{apply}\PY{p}{(}\PY{k}{lambda} \PY{n}{el} \PY{p}{:} \PY{l+s+s2}{\PYZdq{}}\PY{l+s+s2}{=}\PY{l+s+s2}{\PYZdq{}} \PY{o+ow}{in} \PY{n}{el}\PY{p}{)} \PY{c+c1}{\PYZsh{} Almost all categorical variables have a dictionary in the form of \PYZdq{}ans x = y\PYZdq{}}
\PY{n}{wgm\PYZus{}dic}\PY{p}{[}\PY{l+s+s2}{\PYZdq{}}\PY{l+s+s2}{Categorical Ans}\PY{l+s+s2}{\PYZdq{}}\PY{p}{]} \PY{o}{=} \PY{n}{is\PYZus{}category}
\PY{n}{wgm\PYZus{}dic}\PY{o}{.}\PY{n}{loc}\PY{p}{[}\PY{p}{[}\PY{l+s+s2}{\PYZdq{}}\PY{l+s+s2}{Age}\PY{l+s+s2}{\PYZdq{}}\PY{p}{]}\PY{p}{,}\PY{p}{[}\PY{l+s+s2}{\PYZdq{}}\PY{l+s+s2}{Categorical Ans}\PY{l+s+s2}{\PYZdq{}}\PY{p}{]}\PY{p}{]} \PY{o}{=} \PY{k+kc}{False} \PY{c+c1}{\PYZsh{} Manually removing Age}

\PY{n}{wgm\PYZus{}dic}\PY{o}{.}\PY{n}{head}\PY{p}{(}\PY{p}{)}
\PY{c+c1}{\PYZsh{} print(wgm\PYZus{}dic.loc[is\PYZus{}category, [\PYZdq{}Ans dic\PYZdq{}]])}
\PY{c+c1}{\PYZsh{} print(wgm\PYZus{}dic.loc[wgm\PYZus{}dic[\PYZdq{}Categorical Ans\PYZdq{}] == False, [\PYZdq{}Ans dic\PYZdq{}]])}
\end{Verbatim}
\end{tcolorbox}

            \begin{tcolorbox}[breakable, size=fbox, boxrule=.5pt, pad at break*=1mm, opacityfill=0]
\prompt{Out}{outcolor}{11}{\boxspacing}
\begin{Verbatim}[commandchars=\\\{\}]
                        Code  \textbackslash{}
Code\_i
WP5                      WP5
wgt                      wgt
PROJWT                PROJWT
FIELD\_DATE        FIELD\_DATE
YEAR\_CALENDAR  YEAR\_CALENDAR

                                                   Long question  \textbackslash{}
Code\_i
WP5                                                      Country
wgt            National weight, for analysis at the country l{\ldots}
PROJWT         Population weight (included factor to project {\ldots}
FIELD\_DATE                                 Study Completion Date
YEAR\_CALENDAR                                    Year of survey

                Short question  Trust in science value  \textbackslash{}
Code\_i
WP5                    Country                       0
wgt                 Nat weight                       0
PROJWT              Pop weight                       0
FIELD\_DATE     Completion Date                       0
YEAR\_CALENDAR      Survey Year                       0

                                                         Ans dic  \textbackslash{}
Code\_i
WP5            1=United States, 2=Egypt, 3=Morocco, 4=Lebanon{\ldots}
wgt                                      Scale (value of weight)
PROJWT                                   Scale (value of weight)
FIELD\_DATE                                                  Date
YEAR\_CALENDAR                                               Year

               Categorical Ans
Code\_i
WP5                       True
wgt                      False
PROJWT                   False
FIELD\_DATE               False
YEAR\_CALENDAR            False
\end{Verbatim}
\end{tcolorbox}
        
    \begin{tcolorbox}[breakable, size=fbox, boxrule=1pt, pad at break*=1mm,colback=cellbackground, colframe=cellborder]
\prompt{In}{incolor}{ }{\boxspacing}
\begin{Verbatim}[commandchars=\\\{\}]

\end{Verbatim}
\end{tcolorbox}

    \hypertarget{define-functions-acting-on-the-dictionary}{%
\section{Define functions acting on the
dictionary}\label{define-functions-acting-on-the-dictionary}}

    As we will have three different dataframes in three different format
(boolean, numeric and labels) here we define several functions to
``translate'' questions or answers from one dataframe to the others

    \begin{tcolorbox}[breakable, size=fbox, boxrule=1pt, pad at break*=1mm,colback=cellbackground, colframe=cellborder]
\prompt{In}{incolor}{ }{\boxspacing}
\begin{Verbatim}[commandchars=\\\{\}]

\end{Verbatim}
\end{tcolorbox}

    Check if an element is in the series

    \begin{tcolorbox}[breakable, size=fbox, boxrule=1pt, pad at break*=1mm,colback=cellbackground, colframe=cellborder]
\prompt{In}{incolor}{17}{\boxspacing}
\begin{Verbatim}[commandchars=\\\{\}]
\PY{k}{def} \PY{n+nf}{is\PYZus{}in}\PY{p}{(}\PY{n}{series}\PY{p}{,}\PY{n}{element}\PY{p}{)}\PY{p}{:}
    \PY{c+c1}{\PYZsh{}Checks if the element is in the series. If so, it also returns the index of where it is found}
    \PY{k}{try}\PY{p}{:}
        \PY{n}{ind} \PY{o}{=} \PY{n}{series}\PY{p}{[}\PY{n}{series} \PY{o}{==} \PY{n}{element}\PY{p}{]}\PY{o}{.}\PY{n}{index}\PY{p}{[}\PY{l+m+mi}{0}\PY{p}{]}
        \PY{n}{out} \PY{o}{=} \PY{p}{[}\PY{k+kc}{True}\PY{p}{,} \PY{n}{ind}\PY{p}{]}
    \PY{k}{except}\PY{p}{:}
        \PY{n}{out} \PY{o}{=} \PY{p}{[}\PY{k+kc}{False}\PY{p}{,} \PY{k+kc}{None}\PY{p}{]}
    \PY{k}{return} \PY{n}{out}
\end{Verbatim}
\end{tcolorbox}

    \begin{tcolorbox}[breakable, size=fbox, boxrule=1pt, pad at break*=1mm,colback=cellbackground, colframe=cellborder]
\prompt{In}{incolor}{ }{\boxspacing}
\begin{Verbatim}[commandchars=\\\{\}]

\end{Verbatim}
\end{tcolorbox}

    \hypertarget{translate-questions}{%
\subsubsection{Translate questions}\label{translate-questions}}

    Find the index of a question (in format string) from the dictionary
(\texttt{wgm\_dic})

    \begin{tcolorbox}[breakable, size=fbox, boxrule=1pt, pad at break*=1mm,colback=cellbackground, colframe=cellborder]
\prompt{In}{incolor}{18}{\boxspacing}
\begin{Verbatim}[commandchars=\\\{\}]
\PY{k}{def} \PY{n+nf}{find\PYZus{}question\PYZus{}index}\PY{p}{(}\PY{n}{questions}\PY{p}{,} \PY{n}{in\PYZus{}format}\PY{o}{=}\PY{l+s+s2}{\PYZdq{}}\PY{l+s+s2}{Auto}\PY{l+s+s2}{\PYZdq{}}\PY{p}{,} \PY{n}{out\PYZus{}format}\PY{o}{=}\PY{l+s+s2}{\PYZdq{}}\PY{l+s+s2}{Short}\PY{l+s+s2}{\PYZdq{}}\PY{p}{)}\PY{p}{:}
    \PY{c+c1}{\PYZsh{} the question should be a string}
    
    \PY{n}{codes} \PY{o}{=} \PY{n}{wgm\PYZus{}dic}\PY{p}{[}\PY{l+s+s2}{\PYZdq{}}\PY{l+s+s2}{Code}\PY{l+s+s2}{\PYZdq{}}\PY{p}{]}
    \PY{n}{long} \PY{o}{=} \PY{n}{wgm\PYZus{}dic}\PY{p}{[}\PY{l+s+s2}{\PYZdq{}}\PY{l+s+s2}{Long question}\PY{l+s+s2}{\PYZdq{}}\PY{p}{]}
    \PY{n}{short} \PY{o}{=} \PY{n}{wgm\PYZus{}dic}\PY{p}{[}\PY{l+s+s2}{\PYZdq{}}\PY{l+s+s2}{Short question}\PY{l+s+s2}{\PYZdq{}}\PY{p}{]}
    
    \PY{k}{if} \PY{n+nb}{type}\PY{p}{(}\PY{n}{questions}\PY{p}{)} \PY{o}{==} \PY{n+nb}{type}\PY{p}{(}\PY{l+s+s1}{\PYZsq{}}\PY{l+s+s1}{abc}\PY{l+s+s1}{\PYZsq{}}\PY{p}{)}\PY{p}{:} \PY{c+c1}{\PYZsh{} if it\PYZsq{}s a string}
        
        \PY{n}{isincode} \PY{o}{=} \PY{n}{is\PYZus{}in}\PY{p}{(}\PY{n}{codes}\PY{p}{,}\PY{n}{questions}\PY{p}{)}
        \PY{n}{isinlong} \PY{o}{=} \PY{n}{is\PYZus{}in}\PY{p}{(}\PY{n}{long}\PY{p}{,}\PY{n}{questions}\PY{p}{)}
        \PY{n}{isinshort} \PY{o}{=} \PY{n}{is\PYZus{}in}\PY{p}{(}\PY{n}{short}\PY{p}{,}\PY{n}{questions}\PY{p}{)}
        
        \PY{k}{if} \PY{n}{in\PYZus{}format} \PY{o}{==} \PY{l+s+s2}{\PYZdq{}}\PY{l+s+s2}{Auto}\PY{l+s+s2}{\PYZdq{}}\PY{p}{:}
            \PY{k}{if} \PY{n}{isincode}\PY{p}{[}\PY{l+m+mi}{0}\PY{p}{]}\PY{p}{:} \PY{c+c1}{\PYZsh{} if it\PYZsq{}s a code}
                \PY{n}{ind} \PY{o}{=} \PY{n}{isincode}\PY{p}{[}\PY{l+m+mi}{1}\PY{p}{]}
            \PY{k}{elif} \PY{n}{isinlong}\PY{p}{[}\PY{l+m+mi}{0}\PY{p}{]}\PY{p}{:} \PY{c+c1}{\PYZsh{} if it\PYZsq{}s a long}
                \PY{n}{ind} \PY{o}{=} \PY{n}{isinlong}\PY{p}{[}\PY{l+m+mi}{1}\PY{p}{]}
            \PY{k}{elif} \PY{n}{isinshort}\PY{p}{[}\PY{l+m+mi}{0}\PY{p}{]}\PY{p}{:} \PY{c+c1}{\PYZsh{} if it\PYZsq{}s a short}
                \PY{n}{ind} \PY{o}{=} \PY{n}{isinshort}\PY{p}{[}\PY{l+m+mi}{1}\PY{p}{]}
            \PY{k}{else}\PY{p}{:}
                \PY{k}{raise} \PY{n+ne}{Exception}\PY{p}{(}\PY{l+s+s2}{\PYZdq{}}\PY{l+s+s2}{Question not found in any type!}\PY{l+s+s2}{\PYZdq{}}\PY{p}{)} 
            
        \PY{k}{elif} \PY{n}{in\PYZus{}format} \PY{o}{==} \PY{l+s+s2}{\PYZdq{}}\PY{l+s+s2}{Code}\PY{l+s+s2}{\PYZdq{}}\PY{p}{:}
            \PY{k}{if} \PY{n}{isincode}\PY{p}{[}\PY{l+m+mi}{0}\PY{p}{]}\PY{p}{:} \PY{c+c1}{\PYZsh{} if it\PYZsq{}s a code}
                \PY{n}{ind} \PY{o}{=} \PY{n}{isincode}\PY{p}{[}\PY{l+m+mi}{1}\PY{p}{]}
            \PY{k}{else}\PY{p}{:}
                \PY{k}{raise} \PY{n+ne}{Exception}\PY{p}{(}\PY{l+s+s2}{\PYZdq{}}\PY{l+s+s2}{Question not found in the specified type!}\PY{l+s+s2}{\PYZdq{}}\PY{p}{)} 

        \PY{k}{elif} \PY{n}{in\PYZus{}format}\PY{o}{==}\PY{l+s+s2}{\PYZdq{}}\PY{l+s+s2}{Short}\PY{l+s+s2}{\PYZdq{}}\PY{p}{:}
            \PY{k}{if} \PY{n}{isinshort}\PY{p}{[}\PY{l+m+mi}{0}\PY{p}{]}\PY{p}{:} \PY{c+c1}{\PYZsh{} if it\PYZsq{}s a code}
                \PY{n}{ind} \PY{o}{=} \PY{n}{isinshort}\PY{p}{[}\PY{l+m+mi}{1}\PY{p}{]}
            \PY{k}{else}\PY{p}{:}
                \PY{k}{raise} \PY{n+ne}{Exception}\PY{p}{(}\PY{l+s+s2}{\PYZdq{}}\PY{l+s+s2}{Question not found in the specified type!}\PY{l+s+s2}{\PYZdq{}}\PY{p}{)} 

        \PY{k}{elif} \PY{n}{in\PYZus{}format}\PY{o}{==}\PY{l+s+s2}{\PYZdq{}}\PY{l+s+s2}{Long}\PY{l+s+s2}{\PYZdq{}}\PY{p}{:}
            \PY{k}{if} \PY{n}{isinlong}\PY{p}{[}\PY{l+m+mi}{0}\PY{p}{]}\PY{p}{:} \PY{c+c1}{\PYZsh{} if it\PYZsq{}s a code}
                \PY{n}{ind} \PY{o}{=} \PY{n}{isinlong}\PY{p}{[}\PY{l+m+mi}{1}\PY{p}{]}
            \PY{k}{else}\PY{p}{:}
                \PY{k}{raise} \PY{n+ne}{Exception}\PY{p}{(}\PY{l+s+s2}{\PYZdq{}}\PY{l+s+s2}{Question not found in the specified type!}\PY{l+s+s2}{\PYZdq{}}\PY{p}{)} 

        \PY{k}{else}\PY{p}{:}
            \PY{k}{raise} \PY{n+ne}{Exception}\PY{p}{(}\PY{l+s+s2}{\PYZdq{}}\PY{l+s+s2}{Input data type not recognized}\PY{l+s+s2}{\PYZdq{}}\PY{p}{)} 
    \PY{k}{else}\PY{p}{:}
        \PY{k}{raise} \PY{n+ne}{Exception}\PY{p}{(}\PY{l+s+s2}{\PYZdq{}}\PY{l+s+s2}{Invalid question type}\PY{l+s+s2}{\PYZdq{}}\PY{p}{)} 
        
    \PY{k}{return} \PY{n}{ind}
\end{Verbatim}
\end{tcolorbox}

    \begin{tcolorbox}[breakable, size=fbox, boxrule=1pt, pad at break*=1mm,colback=cellbackground, colframe=cellborder]
\prompt{In}{incolor}{ }{\boxspacing}
\begin{Verbatim}[commandchars=\\\{\}]

\end{Verbatim}
\end{tcolorbox}

    Translate the questions (either a string or a list of strings) into any
other format (short, long or code)

    \begin{tcolorbox}[breakable, size=fbox, boxrule=1pt, pad at break*=1mm,colback=cellbackground, colframe=cellborder]
\prompt{In}{incolor}{19}{\boxspacing}
\begin{Verbatim}[commandchars=\\\{\}]
\PY{k}{def} \PY{n+nf}{tanslateQuest}\PY{p}{(}\PY{n}{questions}\PY{p}{,} \PY{n}{in\PYZus{}format}\PY{o}{=}\PY{l+s+s2}{\PYZdq{}}\PY{l+s+s2}{Auto}\PY{l+s+s2}{\PYZdq{}}\PY{p}{,} \PY{n}{out\PYZus{}format}\PY{o}{=}\PY{l+s+s2}{\PYZdq{}}\PY{l+s+s2}{Short}\PY{l+s+s2}{\PYZdq{}}\PY{p}{)}\PY{p}{:}
    \PY{c+c1}{\PYZsh{} Translates a question from a format to another (Only Short, Long or Code)}
    
    \PY{c+c1}{\PYZsh{} questions should be either a list of strings or a string}
    \PY{c+c1}{\PYZsh{} The format can be only Long, Short or Code}
    
    \PY{n}{codes} \PY{o}{=} \PY{n}{wgm\PYZus{}dic}\PY{p}{[}\PY{l+s+s2}{\PYZdq{}}\PY{l+s+s2}{Code}\PY{l+s+s2}{\PYZdq{}}\PY{p}{]}
    \PY{n}{long} \PY{o}{=} \PY{n}{wgm\PYZus{}dic}\PY{p}{[}\PY{l+s+s2}{\PYZdq{}}\PY{l+s+s2}{Long question}\PY{l+s+s2}{\PYZdq{}}\PY{p}{]}
    \PY{n}{short} \PY{o}{=} \PY{n}{wgm\PYZus{}dic}\PY{p}{[}\PY{l+s+s2}{\PYZdq{}}\PY{l+s+s2}{Short question}\PY{l+s+s2}{\PYZdq{}}\PY{p}{]}
    
    \PY{k}{if} \PY{n+nb}{type}\PY{p}{(}\PY{n}{questions}\PY{p}{)} \PY{o}{==} \PY{n+nb}{type}\PY{p}{(}\PY{l+s+s1}{\PYZsq{}}\PY{l+s+s1}{abc}\PY{l+s+s1}{\PYZsq{}}\PY{p}{)}\PY{p}{:} \PY{c+c1}{\PYZsh{} if it\PYZsq{}s a string}
        \PY{n}{questions} \PY{o}{=} \PY{p}{[}\PY{n}{questions}\PY{p}{]} \PY{c+c1}{\PYZsh{} make it as list}
        
    \PY{n}{ind\PYZus{}vec} \PY{o}{=} \PY{n+nb}{list}\PY{p}{(}\PY{p}{)}
    \PY{n}{out\PYZus{}vec} \PY{o}{=} \PY{n+nb}{list}\PY{p}{(}\PY{p}{)}
        
    \PY{k}{for} \PY{n}{quest} \PY{o+ow}{in} \PY{n}{questions}\PY{p}{:}
        \PY{n}{ind} \PY{o}{=} \PY{n}{find\PYZus{}question\PYZus{}index}\PY{p}{(}\PY{n}{quest}\PY{p}{,} \PY{n}{in\PYZus{}format}\PY{o}{=}\PY{l+s+s2}{\PYZdq{}}\PY{l+s+s2}{Auto}\PY{l+s+s2}{\PYZdq{}}\PY{p}{,} \PY{n}{out\PYZus{}format}\PY{o}{=}\PY{l+s+s2}{\PYZdq{}}\PY{l+s+s2}{Short}\PY{l+s+s2}{\PYZdq{}}\PY{p}{)}
        \PY{n}{ind\PYZus{}vec}\PY{o}{.}\PY{n}{append}\PY{p}{(}\PY{n}{ind}\PY{p}{)}
        
        \PY{k}{if} \PY{n}{out\PYZus{}format} \PY{o}{==} \PY{l+s+s2}{\PYZdq{}}\PY{l+s+s2}{Code}\PY{l+s+s2}{\PYZdq{}}\PY{p}{:}
            \PY{n}{out} \PY{o}{=} \PY{n}{codes}\PY{p}{[}\PY{n}{ind}\PY{p}{]}
            \PY{n}{out\PYZus{}vec}\PY{o}{.}\PY{n}{append}\PY{p}{(}\PY{n}{out}\PY{p}{)}
            
        \PY{k}{elif} \PY{n}{out\PYZus{}format} \PY{o}{==} \PY{l+s+s2}{\PYZdq{}}\PY{l+s+s2}{Short}\PY{l+s+s2}{\PYZdq{}}\PY{p}{:}
            \PY{n}{out} \PY{o}{=} \PY{n}{short}\PY{p}{[}\PY{n}{ind}\PY{p}{]}
            \PY{n}{out\PYZus{}vec}\PY{o}{.}\PY{n}{append}\PY{p}{(}\PY{n}{out}\PY{p}{)}
            
        \PY{k}{elif} \PY{n}{out\PYZus{}format} \PY{o}{==} \PY{l+s+s2}{\PYZdq{}}\PY{l+s+s2}{Long}\PY{l+s+s2}{\PYZdq{}}\PY{p}{:}
            \PY{n}{out} \PY{o}{=} \PY{n}{long}\PY{p}{[}\PY{n}{ind}\PY{p}{]}
            \PY{n}{out\PYZus{}vec}\PY{o}{.}\PY{n}{append}\PY{p}{(}\PY{n}{out}\PY{p}{)}
        
        \PY{k}{else}\PY{p}{:}
            \PY{k}{raise} \PY{n+ne}{Exception}\PY{p}{(}\PY{l+s+s2}{\PYZdq{}}\PY{l+s+s2}{Output format not recognized!}\PY{l+s+s2}{\PYZdq{}}\PY{p}{)}
    
    \PY{k}{return} \PY{p}{[}\PY{n}{out\PYZus{}vec}\PY{p}{,} \PY{n}{ind\PYZus{}vec}\PY{p}{]}
\end{Verbatim}
\end{tcolorbox}

    \begin{tcolorbox}[breakable, size=fbox, boxrule=1pt, pad at break*=1mm,colback=cellbackground, colframe=cellborder]
\prompt{In}{incolor}{ }{\boxspacing}
\begin{Verbatim}[commandchars=\\\{\}]

\end{Verbatim}
\end{tcolorbox}

    \hypertarget{tranlsate-answers}{%
\subsubsection{Tranlsate answers}\label{tranlsate-answers}}

    You enter a question and it gives out the possible answers as dictionary
type. Actually the real output is:

\texttt{{[}numNval\_dict,\ num2val,\ val2num{]}}

where \texttt{numNval\_dict} is the dictionary in both diretions (both
num2val and val2num)

    \begin{tcolorbox}[breakable, size=fbox, boxrule=1pt, pad at break*=1mm,colback=cellbackground, colframe=cellborder]
\prompt{In}{incolor}{20}{\boxspacing}
\begin{Verbatim}[commandchars=\\\{\}]
\PY{k}{def} \PY{n+nf}{extractAns}\PY{p}{(}\PY{n}{question}\PY{p}{,} \PY{n}{question\PYZus{}in\PYZus{}format}\PY{o}{=}\PY{l+s+s2}{\PYZdq{}}\PY{l+s+s2}{Auto}\PY{l+s+s2}{\PYZdq{}}\PY{p}{,} \PY{n}{question\PYZus{}out\PYZus{}format}\PY{o}{=}\PY{l+s+s2}{\PYZdq{}}\PY{l+s+s2}{Short}\PY{l+s+s2}{\PYZdq{}}\PY{p}{,} \PY{n}{ans\PYZus{}out\PYZus{}format}\PY{o}{=}\PY{l+s+s2}{\PYZdq{}}\PY{l+s+s2}{AShort}\PY{l+s+s2}{\PYZdq{}}\PY{p}{)}\PY{p}{:}
    \PY{c+c1}{\PYZsh{} you can use only one question}
    \PY{c+c1}{\PYZsh{} Answers can be a list}
    
    \PY{n}{quest\PYZus{}index} \PY{o}{=} \PY{n}{tanslateQuest}\PY{p}{(}\PY{n}{question}\PY{p}{,} \PY{n}{in\PYZus{}format}\PY{o}{=}\PY{n}{question\PYZus{}in\PYZus{}format}\PY{p}{,} \PY{n}{out\PYZus{}format}\PY{o}{=}\PY{l+s+s2}{\PYZdq{}}\PY{l+s+s2}{Code}\PY{l+s+s2}{\PYZdq{}}\PY{p}{)}\PY{p}{[}\PY{l+m+mi}{0}\PY{p}{]}\PY{p}{[}\PY{l+m+mi}{0}\PY{p}{]}
    
    \PY{n}{raw\PYZus{}dict} \PY{o}{=} \PY{n}{wgm\PYZus{}dic}\PY{o}{.}\PY{n}{loc}\PY{p}{[}\PY{p}{[}\PY{n}{quest\PYZus{}index}\PY{p}{]}\PY{p}{,} \PY{p}{[}\PY{l+s+s2}{\PYZdq{}}\PY{l+s+s2}{Ans dic}\PY{l+s+s2}{\PYZdq{}}\PY{p}{]}\PY{p}{]}
    
    \PY{n}{raw\PYZus{}dict} \PY{o}{=} \PY{n}{raw\PYZus{}dict}\PY{o}{.}\PY{n}{values}\PY{p}{[}\PY{l+m+mi}{0}\PY{p}{]}\PY{p}{[}\PY{l+m+mi}{0}\PY{p}{]}
    
    \PY{n}{splitted} \PY{o}{=} \PY{n}{raw\PYZus{}dict}\PY{o}{.}\PY{n}{split}\PY{p}{(}\PY{n}{sep}\PY{o}{=}\PY{l+s+s1}{\PYZsq{}}\PY{l+s+s1}{, }\PY{l+s+s1}{\PYZsq{}}\PY{p}{)}
\PY{c+c1}{\PYZsh{}     print(splitted)}

    \PY{n}{num2val} \PY{o}{=} \PY{n+nb}{dict}\PY{p}{(}\PY{p}{)}
    \PY{n}{val2num} \PY{o}{=} \PY{n+nb}{dict}\PY{p}{(}\PY{p}{)}
    \PY{n}{numNval\PYZus{}dict} \PY{o}{=} \PY{n+nb}{dict}\PY{p}{(}\PY{p}{)}

    \PY{k}{for} \PY{n}{el} \PY{o+ow}{in} \PY{n}{splitted}\PY{p}{:}
        \PY{k}{if} \PY{n+nb}{len}\PY{p}{(}\PY{n}{el}\PY{p}{)}\PY{o}{\PYZlt{}}\PY{l+m+mi}{3}\PY{p}{:}
            \PY{k}{continue}
        
\PY{c+c1}{\PYZsh{}         print(el)}
        \PY{p}{[}\PY{n}{num}\PY{p}{,} \PY{n}{val}\PY{p}{]} \PY{o}{=}\PY{n}{el}\PY{o}{.}\PY{n}{split}\PY{p}{(}\PY{n}{sep}\PY{o}{=}\PY{l+s+s1}{\PYZsq{}}\PY{l+s+s1}{=}\PY{l+s+s1}{\PYZsq{}}\PY{p}{)}
        \PY{n}{num} \PY{o}{=} \PY{n+nb}{int}\PY{p}{(}\PY{n}{num}\PY{p}{)}

        \PY{n}{num2val}\PY{p}{[}\PY{n}{num}\PY{p}{]} \PY{o}{=} \PY{n}{val}
        \PY{n}{numNval\PYZus{}dict}\PY{p}{[}\PY{n}{num}\PY{p}{]} \PY{o}{=} \PY{n}{val}

        \PY{n}{val2num}\PY{p}{[}\PY{n}{val}\PY{p}{]} \PY{o}{=} \PY{n}{num}
        \PY{n}{numNval\PYZus{}dict}\PY{p}{[}\PY{n}{val}\PY{p}{]} \PY{o}{=} \PY{n}{num}
    
    \PY{k}{return} \PY{p}{[}\PY{n}{numNval\PYZus{}dict}\PY{p}{,} \PY{n}{num2val}\PY{p}{,} \PY{n}{val2num}\PY{p}{]}
\end{Verbatim}
\end{tcolorbox}

    \begin{tcolorbox}[breakable, size=fbox, boxrule=1pt, pad at break*=1mm,colback=cellbackground, colframe=cellborder]
\prompt{In}{incolor}{ }{\boxspacing}
\begin{Verbatim}[commandchars=\\\{\}]

\end{Verbatim}
\end{tcolorbox}

    Translate your answes from one format to the other (you need to specify
the question, of course)

    \begin{tcolorbox}[breakable, size=fbox, boxrule=1pt, pad at break*=1mm,colback=cellbackground, colframe=cellborder]
\prompt{In}{incolor}{21}{\boxspacing}
\begin{Verbatim}[commandchars=\\\{\}]
\PY{k}{def} \PY{n+nf}{translateAns}\PY{p}{(}\PY{n}{question}\PY{p}{,} \PY{n}{answers}\PY{p}{,} \PY{n}{question\PYZus{}in\PYZus{}format}\PY{o}{=}\PY{l+s+s2}{\PYZdq{}}\PY{l+s+s2}{Auto}\PY{l+s+s2}{\PYZdq{}}\PY{p}{,} \PY{n}{question\PYZus{}out\PYZus{}format}\PY{o}{=}\PY{l+s+s2}{\PYZdq{}}\PY{l+s+s2}{Short}\PY{l+s+s2}{\PYZdq{}}\PY{p}{,} \PY{n}{ans\PYZus{}out\PYZus{}format}\PY{o}{=}\PY{l+s+s2}{\PYZdq{}}\PY{l+s+s2}{Auto}\PY{l+s+s2}{\PYZdq{}}\PY{p}{)}\PY{p}{:}
    \PY{c+c1}{\PYZsh{} you can use only one question}
    \PY{c+c1}{\PYZsh{} Answers can be a list}
    \PY{c+c1}{\PYZsh{} At the moment ans\PYZus{}out\PYZus{}format can be only Auto}
    
    \PY{n}{trans\PYZus{}Ans\PYZus{}dict} \PY{o}{=} \PY{n}{extractAns}\PY{p}{(}\PY{n}{question}\PY{p}{,} \PY{n}{question\PYZus{}in\PYZus{}format}\PY{o}{=}\PY{l+s+s2}{\PYZdq{}}\PY{l+s+s2}{Auto}\PY{l+s+s2}{\PYZdq{}}\PY{p}{,} \PY{n}{question\PYZus{}out\PYZus{}format}\PY{o}{=}\PY{l+s+s2}{\PYZdq{}}\PY{l+s+s2}{Short}\PY{l+s+s2}{\PYZdq{}}\PY{p}{,} \PY{n}{ans\PYZus{}out\PYZus{}format}\PY{o}{=}\PY{l+s+s2}{\PYZdq{}}\PY{l+s+s2}{AShort}\PY{l+s+s2}{\PYZdq{}}\PY{p}{)}\PY{p}{[}\PY{l+m+mi}{0}\PY{p}{]}
    
    \PY{k}{if} \PY{o+ow}{not} \PY{n+nb}{type}\PY{p}{(}\PY{n}{answers}\PY{p}{)}\PY{o}{==}\PY{n+nb}{type}\PY{p}{(}\PY{n+nb}{list}\PY{p}{(}\PY{p}{)}\PY{p}{)}\PY{p}{:} \PY{c+c1}{\PYZsh{} Turn the answers in a list, so we can iterate}
        \PY{n}{answers} \PY{o}{=} \PY{p}{[}\PY{n}{answers}\PY{p}{]}
        
    \PY{n}{translated\PYZus{}ans} \PY{o}{=} \PY{n+nb}{list}\PY{p}{(}\PY{p}{)}
    \PY{k}{for} \PY{n}{ans} \PY{o+ow}{in} \PY{n}{answers}\PY{p}{:}
        \PY{n}{strans\PYZus{}ans} \PY{o}{=} \PY{n}{trans\PYZus{}Ans\PYZus{}dict}\PY{p}{[}\PY{n}{ans}\PY{p}{]}
        \PY{n}{translated\PYZus{}ans}\PY{o}{.}\PY{n}{append}\PY{p}{(}\PY{n}{strans\PYZus{}ans}\PY{p}{)}
        
    \PY{k}{if} \PY{n+nb}{len}\PY{p}{(}\PY{n}{translated\PYZus{}ans}\PY{p}{)} \PY{o}{==} \PY{l+m+mi}{1}\PY{p}{:} 
        \PY{n}{translated\PYZus{}ans} \PY{o}{=} \PY{n}{translated\PYZus{}ans}\PY{p}{[}\PY{l+m+mi}{0}\PY{p}{]}
        
    \PY{k}{return} \PY{n}{translated\PYZus{}ans}
        
        
\PY{c+c1}{\PYZsh{}     quest\PYZus{}index = tanslateQuest(question, in\PYZus{}format=question\PYZus{}in\PYZus{}format, out\PYZus{}format=\PYZdq{}Code\PYZdq{})[0]}
    
\PY{c+c1}{\PYZsh{}     raw\PYZus{}dict = wgm\PYZus{}dic.loc[[quest\PYZus{}index], [\PYZdq{}Ans dic\PYZdq{}]]}
    
    \PY{k}{return} \PY{n}{raw\PYZus{}dict}
\end{Verbatim}
\end{tcolorbox}

    \begin{tcolorbox}[breakable, size=fbox, boxrule=1pt, pad at break*=1mm,colback=cellbackground, colframe=cellborder]
\prompt{In}{incolor}{ }{\boxspacing}
\begin{Verbatim}[commandchars=\\\{\}]

\end{Verbatim}
\end{tcolorbox}

    \hypertarget{clean-the-labels-in-the-database}{%
\section{Clean the labels in the
database}\label{clean-the-labels-in-the-database}}

    Make a dictionary of all the indeces -\textgreater{} \texttt{index\_dic}

    \begin{tcolorbox}[breakable, size=fbox, boxrule=1pt, pad at break*=1mm,colback=cellbackground, colframe=cellborder]
\prompt{In}{incolor}{24}{\boxspacing}
\begin{Verbatim}[commandchars=\\\{\}]
\PY{c+c1}{\PYZsh{} need to create a dictionary }
\PY{c+c1}{\PYZsh{} wgm\PYZus{}dic[\PYZdq{}Code\PYZdq{}]}

\PY{n}{list\PYZus{}of\PYZus{}codes} \PY{o}{=} \PY{n+nb}{list}\PY{p}{(}\PY{n}{wgm\PYZus{}dic}\PY{p}{[}\PY{l+s+s2}{\PYZdq{}}\PY{l+s+s2}{Code}\PY{l+s+s2}{\PYZdq{}}\PY{p}{]}\PY{p}{)}

\PY{n}{index\PYZus{}dic} \PY{o}{=} \PY{n+nb}{dict}\PY{p}{(}\PY{p}{)}

\PY{k}{for} \PY{n}{code} \PY{o+ow}{in} \PY{n}{list\PYZus{}of\PYZus{}codes}\PY{p}{:}
    \PY{n}{short\PYZus{}vers} \PY{o}{=} \PY{n}{wgm\PYZus{}dic}\PY{o}{.}\PY{n}{loc}\PY{p}{[}\PY{p}{[}\PY{n}{code}\PY{p}{]}\PY{p}{,}\PY{p}{[}\PY{l+s+s2}{\PYZdq{}}\PY{l+s+s2}{Short question}\PY{l+s+s2}{\PYZdq{}}\PY{p}{]}\PY{p}{]}\PY{o}{.}\PY{n}{values}\PY{p}{[}\PY{l+m+mi}{0}\PY{p}{]}\PY{p}{[}\PY{l+m+mi}{0}\PY{p}{]}
    
    \PY{n}{index\PYZus{}dic}\PY{p}{[}\PY{n}{code}\PY{p}{]} \PY{o}{=} \PY{n}{short\PYZus{}vers}
    
\PY{c+c1}{\PYZsh{} index\PYZus{}dic}
\end{Verbatim}
\end{tcolorbox}

    \begin{tcolorbox}[breakable, size=fbox, boxrule=1pt, pad at break*=1mm,colback=cellbackground, colframe=cellborder]
\prompt{In}{incolor}{ }{\boxspacing}
\begin{Verbatim}[commandchars=\\\{\}]

\end{Verbatim}
\end{tcolorbox}

    \hypertarget{make-the-numeric-version-of-the-database}{%
\subsubsection{Make the numeric version of the
database}\label{make-the-numeric-version-of-the-database}}

i.e.~columns names (questions) are in version short, while all the
answers are numeric

-\textgreater{} This dataframe will be called \texttt{wgm\_numeric}

    \begin{tcolorbox}[breakable, size=fbox, boxrule=1pt, pad at break*=1mm,colback=cellbackground, colframe=cellborder]
\prompt{In}{incolor}{27}{\boxspacing}
\begin{Verbatim}[commandchars=\\\{\}]
\PY{n}{wgm\PYZus{}numeric} \PY{o}{=} \PY{n}{wgm}\PY{o}{.}\PY{n}{rename}\PY{p}{(}\PY{n}{columns}\PY{o}{=}\PY{n}{index\PYZus{}dic}\PY{p}{)}
\PY{n}{wgm\PYZus{}numeric}\PY{o}{.}\PY{n}{head}\PY{p}{(}\PY{p}{)}
\end{Verbatim}
\end{tcolorbox}

            \begin{tcolorbox}[breakable, size=fbox, boxrule=.5pt, pad at break*=1mm, opacityfill=0]
\prompt{Out}{outcolor}{27}{\boxspacing}
\begin{Verbatim}[commandchars=\\\{\}]
   Country  Nat weight     Pop weight Completion Date  Survey Year  \textbackslash{}
0        1    0.652821  171769.597742      2018-01-08         2018
1        1    0.695706  183053.484155      2018-01-08         2018
2        1    0.523829  137829.328857      2018-01-08         2018
3        1    0.764442  201139.215039      2018-01-08         2018
4        1    3.327946  875645.512738      2018-01-08         2018

   Know Science  Understand meaning Sci  Study disease is science  \textbackslash{}
0             3                       2                         1
1             2                       2                         1
2             2                       2                         1
3             2                       1                         1
4             2                       1                         1

   Poetry is science  Learned Sci in Prim.School  {\ldots} Age Pers Age Coho  \textbackslash{}
0                  2                           2  {\ldots}       72        3
1                  2                           1  {\ldots}       72        3
2                 98                           1  {\ldots}       85        3
3                  2                           1  {\ldots}       54        3
4                  2                           1  {\ldots}       20        1

   Gender  Education  Area Type  Income  Region  Subjective Income  \textbackslash{}
0       2          3          1       3       7                  2
1       1          2          2       3       7                  1
2       1          2          1       2       7                  3
3       1          3          2       5       7                  1
4       1          2          2       2       7                  1

   Income Level Employment
0             4          6
1             4          6
2             4          6
3             4          1
4             4          6

[5 rows x 60 columns]
\end{Verbatim}
\end{tcolorbox}
        
    \begin{tcolorbox}[breakable, size=fbox, boxrule=1pt, pad at break*=1mm,colback=cellbackground, colframe=cellborder]
\prompt{In}{incolor}{ }{\boxspacing}
\begin{Verbatim}[commandchars=\\\{\}]

\end{Verbatim}
\end{tcolorbox}

    \hypertarget{make-the-version-with-labels-of-the-database}{%
\subsubsection{Make the version with labels of the
database}\label{make-the-version-with-labels-of-the-database}}

i.e.~questions/columns as short and answers as val (not numeric)

-\textgreater{} \texttt{wgm\_labels}

Note: some values are still numeric (such as the age) as it doesn't make
any sense to change it. However, all the categorical questions will be
changed

    \begin{tcolorbox}[breakable, size=fbox, boxrule=1pt, pad at break*=1mm,colback=cellbackground, colframe=cellborder]
\prompt{In}{incolor}{28}{\boxspacing}
\begin{Verbatim}[commandchars=\\\{\}]
\PY{n}{wgm\PYZus{}labels} \PY{o}{=} \PY{n}{pd}\PY{o}{.}\PY{n}{DataFrame}\PY{p}{(}\PY{p}{)} \PY{c+c1}{\PYZsh{} empty df}

\PY{n}{list\PYZus{}of\PYZus{}questions} \PY{o}{=} \PY{n+nb}{list}\PY{p}{(}\PY{n}{wgm\PYZus{}dic}\PY{p}{[}\PY{l+s+s2}{\PYZdq{}}\PY{l+s+s2}{Short question}\PY{l+s+s2}{\PYZdq{}}\PY{p}{]}\PY{p}{)}
\PY{n}{list\PYZus{}of\PYZus{}catheg\PYZus{}questions} \PY{o}{=} \PY{n+nb}{list}\PY{p}{(}\PY{p}{)}

\PY{k}{def} \PY{n+nf}{translate\PYZus{}column}\PY{p}{(}\PY{n}{var}\PY{p}{)}\PY{p}{:}
    \PY{k}{try}\PY{p}{:}
        \PY{n}{out} \PY{o}{=} \PY{n}{ans\PYZus{}dic}\PY{p}{[}\PY{n}{var}\PY{p}{]}
    \PY{k}{except}\PY{p}{:}
        \PY{n}{out} \PY{o}{=} \PY{l+s+s2}{\PYZdq{}}\PY{l+s+s2}{Empty}\PY{l+s+s2}{\PYZdq{}}
    \PY{k}{return} \PY{n}{out}

\PY{k}{for} \PY{n}{quest} \PY{o+ow}{in} \PY{n}{list\PYZus{}of\PYZus{}questions}\PY{p}{:}
    
    \PY{k}{if} \PY{o+ow}{not} \PY{n}{wgm\PYZus{}dic}\PY{o}{.}\PY{n}{loc}\PY{p}{[}\PY{n}{wgm\PYZus{}dic}\PY{p}{[}\PY{l+s+s2}{\PYZdq{}}\PY{l+s+s2}{Short question}\PY{l+s+s2}{\PYZdq{}}\PY{p}{]} \PY{o}{==} \PY{n}{quest}\PY{p}{,} \PY{p}{[}\PY{l+s+s2}{\PYZdq{}}\PY{l+s+s2}{Categorical Ans}\PY{l+s+s2}{\PYZdq{}}\PY{p}{]}\PY{p}{]}\PY{o}{.}\PY{n}{values}\PY{p}{[}\PY{l+m+mi}{0}\PY{p}{]}\PY{p}{[}\PY{l+m+mi}{0}\PY{p}{]}\PY{p}{:} 
        \PY{c+c1}{\PYZsh{} if it\PYZsq{}s not a cathegorical variable}
        \PY{c+c1}{\PYZsh{} Just copy it the way it is}
        \PY{n}{wgm\PYZus{}labels}\PY{p}{[}\PY{n}{quest}\PY{p}{]} \PY{o}{=} \PY{n}{wgm\PYZus{}numeric}\PY{p}{[}\PY{n}{quest}\PY{p}{]}
    \PY{k}{else}\PY{p}{:}
        \PY{n}{list\PYZus{}of\PYZus{}catheg\PYZus{}questions}\PY{o}{.}\PY{n}{append}\PY{p}{(}\PY{n}{quest}\PY{p}{)}
        
        \PY{n}{entire\PYZus{}col} \PY{o}{=} \PY{n}{wgm\PYZus{}numeric}\PY{p}{[}\PY{n}{quest}\PY{p}{]}

        \PY{n}{ans\PYZus{}dic} \PY{o}{=} \PY{n}{extractAns}\PY{p}{(}\PY{n}{quest}\PY{p}{)}\PY{p}{[}\PY{l+m+mi}{0}\PY{p}{]}

        \PY{n}{entire\PYZus{}col\PYZus{}text} \PY{o}{=} \PY{n}{entire\PYZus{}col}\PY{o}{.}\PY{n}{apply}\PY{p}{(}\PY{n}{translate\PYZus{}column}\PY{p}{)}

        \PY{n}{wgm\PYZus{}labels}\PY{p}{[}\PY{n}{quest}\PY{p}{]} \PY{o}{=} \PY{n}{entire\PYZus{}col\PYZus{}text}
    
\end{Verbatim}
\end{tcolorbox}

    \begin{tcolorbox}[breakable, size=fbox, boxrule=1pt, pad at break*=1mm,colback=cellbackground, colframe=cellborder]
\prompt{In}{incolor}{29}{\boxspacing}
\begin{Verbatim}[commandchars=\\\{\}]
\PY{n}{wgm\PYZus{}labels}\PY{o}{.}\PY{n}{head}\PY{p}{(}\PY{p}{)}
\end{Verbatim}
\end{tcolorbox}

            \begin{tcolorbox}[breakable, size=fbox, boxrule=.5pt, pad at break*=1mm, opacityfill=0]
\prompt{Out}{outcolor}{29}{\boxspacing}
\begin{Verbatim}[commandchars=\\\{\}]
         Country  Nat weight     Pop weight Completion Date  Survey Year  \textbackslash{}
0  United States    0.652821  171769.597742      2018-01-08         2018
1  United States    0.695706  183053.484155      2018-01-08         2018
2  United States    0.523829  137829.328857      2018-01-08         2018
3  United States    0.764442  201139.215039      2018-01-08         2018
4  United States    3.327946  875645.512738      2018-01-08         2018

  Know Science Understand meaning Sci Study disease is science  \textbackslash{}
0     Not much             Some of it                      Yes
1         Some             Some of it                      Yes
2         Some             Some of it                      Yes
3         Some              All of it                      Yes
4         Some              All of it                      Yes

  Poetry is science Learned Sci in Prim.School  {\ldots} Age Pers  Age Coho  \textbackslash{}
0                No                         No  {\ldots}       72       50+
1                No                        Yes  {\ldots}       72       50+
2              (DK)                        Yes  {\ldots}       85       50+
3                No                        Yes  {\ldots}       54       50+
4                No                        Yes  {\ldots}       20  15 to 29

   Gender  Education                          Area Type      Income  \textbackslash{}
0  Female   Tertiary  Lives in rural area or small town  Middle 20\%
1    Male  Secondary    Lives in city or suburb of city  Middle 20\%
2    Male  Secondary  Lives in rural area or small town  Second 20\%
3    Male   Tertiary    Lives in city or suburb of city     Top 20\%
4    Male  Secondary    Lives in city or suburb of city  Second 20\%

             Region Income Level  \textbackslash{}
0  Northern America  High income
1  Northern America  High income
2  Northern America  High income
3  Northern America  High income
4  Northern America  High income

                                   Subjective Income  \textbackslash{}
0                       Getting by on present income
1            Living comfortably by on present income
2   Finding it difficult/very difficult to get by{\ldots}
3            Living comfortably by on present income
4            Living comfortably by on present income

                           Employment
0                    Out of workforce
1                    Out of workforce
2                    Out of workforce
3  Employed full time for an employer
4                    Out of workforce

[5 rows x 60 columns]
\end{Verbatim}
\end{tcolorbox}
        
    \begin{tcolorbox}[breakable, size=fbox, boxrule=1pt, pad at break*=1mm,colback=cellbackground, colframe=cellborder]
\prompt{In}{incolor}{ }{\boxspacing}
\begin{Verbatim}[commandchars=\\\{\}]

\end{Verbatim}
\end{tcolorbox}

    \hypertarget{boolean-version-of-the-database-aka-dummy-coded}{%
\section{Boolean version of the database (aka dummy
coded)}\label{boolean-version-of-the-database-aka-dummy-coded}}

i.e.~each columns represents one combination of question and answers
(e.g.~``Vaccines:Trust'') the valus in the cells are then just booleans.
This is useful for performing dichotomous analysis

    \begin{tcolorbox}[breakable, size=fbox, boxrule=1pt, pad at break*=1mm,colback=cellbackground, colframe=cellborder]
\prompt{In}{incolor}{31}{\boxspacing}
\begin{Verbatim}[commandchars=\\\{\}]
\PY{n}{wgm\PYZus{}bool} \PY{o}{=} \PY{n}{pd}\PY{o}{.}\PY{n}{DataFrame}\PY{p}{(}\PY{p}{)}

\PY{n}{list\PYZus{}of\PYZus{}attitudes} \PY{o}{=} \PY{n+nb}{list}\PY{p}{(}\PY{p}{)}

\PY{k}{for} \PY{n}{quest} \PY{o+ow}{in} \PY{n}{list\PYZus{}of\PYZus{}catheg\PYZus{}questions}\PY{p}{:} \PY{c+c1}{\PYZsh{} for each question}
    \PY{n}{num2val\PYZus{}dic} \PY{o}{=} \PY{n}{extractAns}\PY{p}{(}\PY{n}{quest}\PY{p}{)}\PY{p}{[}\PY{l+m+mi}{1}\PY{p}{]}
    
    \PY{k}{for} \PY{n}{key} \PY{o+ow}{in} \PY{n}{num2val\PYZus{}dic}\PY{p}{:} \PY{c+c1}{\PYZsh{} for each anwer}
        \PY{n}{val} \PY{o}{=} \PY{n}{num2val\PYZus{}dic}\PY{p}{[}\PY{n}{key}\PY{p}{]}
        \PY{n}{full\PYZus{}str} \PY{o}{=} \PY{n}{quest}\PY{o}{+}\PY{l+s+s2}{\PYZdq{}}\PY{l+s+s2}{:}\PY{l+s+s2}{\PYZdq{}}\PY{o}{+}\PY{n}{val}
        
        \PY{n}{list\PYZus{}of\PYZus{}attitudes}\PY{o}{.}\PY{n}{append}\PY{p}{(}\PY{n}{full\PYZus{}str}\PY{p}{)}
        
        \PY{n}{col} \PY{o}{=} \PY{n}{wgm\PYZus{}labels}\PY{p}{[}\PY{n}{quest}\PY{p}{]} \PY{o}{==} \PY{n}{val}
        
        \PY{n}{wgm\PYZus{}bool}\PY{p}{[}\PY{n}{full\PYZus{}str}\PY{p}{]} \PY{o}{=} \PY{n}{col}
        
\PY{c+c1}{\PYZsh{} num2val\PYZus{}dic}
\PY{c+c1}{\PYZsh{} full\PYZus{}str}
\end{Verbatim}
\end{tcolorbox}

    \begin{tcolorbox}[breakable, size=fbox, boxrule=1pt, pad at break*=1mm,colback=cellbackground, colframe=cellborder]
\prompt{In}{incolor}{ }{\boxspacing}
\begin{Verbatim}[commandchars=\\\{\}]

\end{Verbatim}
\end{tcolorbox}

    \hypertarget{end-of-cleaning}{%
\section{End of cleaning}\label{end-of-cleaning}}

    Save the files to your favourite format

    \begin{tcolorbox}[breakable, size=fbox, boxrule=1pt, pad at break*=1mm,colback=cellbackground, colframe=cellborder]
\prompt{In}{incolor}{33}{\boxspacing}
\begin{Verbatim}[commandchars=\\\{\}]
\PY{c+c1}{\PYZsh{} Excel}
\PY{c+c1}{\PYZsh{} filename = \PYZdq{}wgm2018\PYZus{}cleaned\PYZdq{}}
\PY{c+c1}{\PYZsh{} Excel\PYZus{}writer = pd.ExcelWriter(filename+\PYZdq{}.xlsx\PYZdq{}, engine = \PYZsq{}xlsxwriter\PYZsq{})}
\PY{c+c1}{\PYZsh{} wgm\PYZus{}dic.to\PYZus{}excel(Excel\PYZus{}writer, sheet\PYZus{}name=\PYZsq{}Dictionary\PYZsq{})}
\PY{c+c1}{\PYZsh{} wgm\PYZus{}numeric.to\PYZus{}excel(Excel\PYZus{}writer, sheet\PYZus{}name=\PYZsq{}Numeric\PYZsq{})}
\PY{c+c1}{\PYZsh{} wgm\PYZus{}labels.to\PYZus{}excel(Excel\PYZus{}writer, sheet\PYZus{}name=\PYZsq{}Labels\PYZsq{})}
\PY{c+c1}{\PYZsh{} wgm\PYZus{}bool.to\PYZus{}excel(Excel\PYZus{}writer, sheet\PYZus{}name=\PYZsq{}Booleans\PYZsq{})}

\PY{c+c1}{\PYZsh{} Pickle}
\PY{n}{basename} \PY{o}{=} \PY{l+s+s2}{\PYZdq{}}\PY{l+s+s2}{wgm2018\PYZus{}clean\PYZus{}}\PY{l+s+s2}{\PYZdq{}}
\PY{n}{wgm\PYZus{}dic}\PY{o}{.}\PY{n}{to\PYZus{}pickle}\PY{p}{(}\PY{n}{basename}\PY{o}{+}\PY{l+s+s2}{\PYZdq{}}\PY{l+s+s2}{dictionary}\PY{l+s+s2}{\PYZdq{}}\PY{o}{+}\PY{l+s+s2}{\PYZdq{}}\PY{l+s+s2}{.pkl}\PY{l+s+s2}{\PYZdq{}}\PY{p}{)}
\PY{n}{wgm\PYZus{}numeric}\PY{o}{.}\PY{n}{to\PYZus{}pickle}\PY{p}{(}\PY{n}{basename}\PY{o}{+}\PY{l+s+s2}{\PYZdq{}}\PY{l+s+s2}{numeric}\PY{l+s+s2}{\PYZdq{}}\PY{o}{+}\PY{l+s+s2}{\PYZdq{}}\PY{l+s+s2}{.pkl}\PY{l+s+s2}{\PYZdq{}}\PY{p}{)}
\PY{n}{wgm\PYZus{}labels}\PY{o}{.}\PY{n}{to\PYZus{}pickle}\PY{p}{(}\PY{n}{basename}\PY{o}{+}\PY{l+s+s2}{\PYZdq{}}\PY{l+s+s2}{labels}\PY{l+s+s2}{\PYZdq{}}\PY{o}{+}\PY{l+s+s2}{\PYZdq{}}\PY{l+s+s2}{.pkl}\PY{l+s+s2}{\PYZdq{}}\PY{p}{)}
\PY{n}{wgm\PYZus{}bool}\PY{o}{.}\PY{n}{to\PYZus{}pickle}\PY{p}{(}\PY{n}{basename}\PY{o}{+}\PY{l+s+s2}{\PYZdq{}}\PY{l+s+s2}{boolean}\PY{l+s+s2}{\PYZdq{}}\PY{o}{+}\PY{l+s+s2}{\PYZdq{}}\PY{l+s+s2}{.pkl}\PY{l+s+s2}{\PYZdq{}}\PY{p}{)}

\PY{c+c1}{\PYZsh{} Read}
\PY{c+c1}{\PYZsh{} print(pd.read\PYZus{}pickle(basename+\PYZdq{}boolean\PYZdq{}+\PYZdq{}.pkl\PYZdq{})}
\end{Verbatim}
\end{tcolorbox}

    The files are now ready to be used in the following codes


    % Add a bibliography block to the postdoc
    
    
    
\end{document}
